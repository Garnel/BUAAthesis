% !Mode:: "TeX:UTF-8"
\begin{cabstract}
随着医学和计算机图像图形学的发展,传统的内窥镜技术正逐渐被可以无损检测的虚拟内窥镜技术所取代。而想要快速准确地进行大肠虚拟漫游,一个首要步骤是提取大肠模型的中心路径以指导视点移动。

本文主要针对大肠中心路径提取算法进行了研究,并且在快速行进法的基础上成功提取了大肠的中心路径。首先将网格数据体素化,然后在3个坐标轴方向对生成的大肠体数据进行切片,对每一个切片通过增强的快速行进法提取二维的骨架,最后将3个轴相对应的三个骨架群求交集得到最终的中心路径。
\end{cabstract}

\begin{eabstract}
With the development of medicine and computer graphics, the traditional endoscope technology is gradually replaced by virtual endoscopy techniques for nondestructive diagnosis. However to perform a virtual roaming in intestine accurately, the first step is to extract the centerline of the intestine model to guide viewpoints.

The paper focuses on 3D centerline of the intestine's extraction algorithm and its implementation based on fast marching method. First, voxelize the mesh data, then, slice the generated volume file of the intestine in three axis directions and skeletonization each slice. At last, intersect the three skeleton sets to get the centerline.
\end{eabstract}
