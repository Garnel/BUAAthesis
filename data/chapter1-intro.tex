% !Mode:: "TeX:UTF-8"
\chapter{绪论}

\section{课题背景与意义}
\subsection{课题来源}
本课题来自09级软件学院毕业设计院系老师提供选题。项目名为基于医学CT成像的人体大肠虚拟漫游,本人负责提取漫游路径。

\subsection{课题背景与意义}
随着人们饮食习惯的改变,肠癌患者比例逐年增加,根据世界卫生组织08年的统计,肠癌死亡率已经是世界排名第三的。

在对大肠疾病的医学检查过程中,由于发病位置的特殊性,传统方法是使用肠镜和钡餐。肠镜检查是用一个配备有摄像头的柔软管子,经人体肛门进入大肠内进行观察。但是这种检查方法,对检查设备的操作技术要求较高,过程复杂且耗费时间,会给病人带来较大的痛苦,不便进行多次重复检查。尤其是对于体质较弱的病人可能形成肠穿孔,并且这种检查方法存在观测的盲区。另一种方法钡餐则是在结肠内填充对比剂——钡,然后通过X线机观察钡的充盈程度来观察大肠息肉,但是这种方法正确性很大程度上依赖于医生的经验。

虚拟内窥镜技术是一种医学影像技术、计算机图像图形学及虚拟现实等学科的交叉与融合而逐步形成的一种独特的技术。它使用CT扫描得到病人的肠道断层图像,然后经过三维重建后通过虚拟漫游来检查大肠内表面的病变。这种方法无须往病人体内插入异物,极大地减轻了病人的痛苦,同一套数据可以重复使用任意多次,同时还能对常规内窥镜无法检查的区域(如椎管等)进行检查,在医疗诊断及手术上有着重要的意义。

虚拟内窥镜的各种技术中,在三维数据中确定漫游路径是当前研究中的难题之一,漫游路径是虚拟相机移动和获取内窥图像的重要基础。目前,最常用的方法是提取结肠的中心线作为漫游路径。医生依此路径控制虚拟相机选择感兴趣的方向察看结肠内部,中心路径的质量直接影响医生的检查结果,高效的中心路径提取算法是漫游效果的保证。

\section{国内外研究现状}
\subsection{虚拟内窥镜技术的发展现状}
随着医学图像处理和三维可视化技术的发展,虚拟内窥镜(Virtual Endoscopy, VE)以其非入侵性、可重复性等明显的优势获得广泛的研究和应用。

自1993年Vining等人首次提出虚拟支气管内窥镜以来,虚拟内窥镜技术已被应用到许多临床实验和各种医学诊断中,主要集中在那些具有空腔组织结构的器官上,如结肠、气管、血管、内耳等。比较典型的应用有美国GE Research \& Development Center开发的虚拟内窥镜医学应用系统(VEMA),可以检查人体的多个部位:虚拟结肠、虚拟支气管和虚拟脉管。West Forest大学虚拟内窥镜研究中心研发了一套Free Flight虚拟内窥镜软件系统。美国Boston Surgic Planning Laboratory建立的一种虚拟耳窥镜系统,以三维形式显示耳的解剖结构来模拟传统内窥镜对内耳的检查过程。法国Laennec Hospital开发的虚拟内窥镜系统主要用来对食管、喉进行虚拟内窥。

国内对虚拟内窥镜技术也十分重视,目前许多重点大学和机构都在研究。其中西北大学的交互式虚拟内窥镜系统、中国科学院自动化所的3DMED系统已初现成效;浙江大学、中国科技大学等高校也有相关的研究。

\subsection{路径规划的发展现状}
路径规划是虚拟漫游的基础,要想快速准确地进行虚拟内窥漫游,首先要对视点(虚拟相机)的移动路径进行规划。路径规划技术的研究内容归纳为如何在器官内部,快速、准确地规划一条或多条连续、平滑的路径,使得器官表面的每个体素尽可能地被路径上的点看到。理想的路径应该保证每个边界体素至少被一条路径上的点检测到。根据路径规划算法的的发展以及不同的漫游方式,路径规划分为以下几种情况:
\begin{description}
    \item[手工路径规划] 操作者通过系统接口,输入每一步需要的路径参数,控制视点在虚拟器官中,一步步前行。 这是一个十分耗时的漫游方式,倘若是一个毫无经验的操作者,可能迷失方向。
  \item[自动路径规划] 在漫游前先提取出漫游的路径,然后系统自动进行漫游,这能一定程度上帮助医生快速了解病变情况。由于缺少交互功能,不能对某一部分进行细致的检查。
  \item[交互式路径规划] 这种方式实现了操作者与系统的互动。交互式路径规划允许操作者指定路径的起点和终点,然后系统会实时地提取导航路径,实现实时漫游功能。在计算机运算能力越来越强的今天,这种交互式的路径规划以其良好的体验已经成为了研究的重点。
\end{description}

中心路径为视点提供开阔的视野,目前的虚拟漫游一般都选用中心路径作为漫游路径,医生依此路径控制视点,选择感兴趣的方向检查器官。中心路径一般用腔体器官的中轴线来表示。常用的中心路径规划算法一般有下面几种:
\begin{description}
    \item[手工标记] \hfill \\
用户在二维切片中手动标定中心点,然后将一系列的中心点进行插值形成中心路径。这种方法费事费力,对操作者要求很高,并且二维切片的中心点并不一定就是三维的中心点,可能存在误差。所以这种方法已经不怎么使用了。
  \item[拓扑细化] \hfill \\
这种方法在不改变器官的拓扑结构的情况下,反复将外层体素剥离,直至剩下单体素。这种方法计算量十分庞大,对于大肠体数据往往得用几个小时来提取中心路径。为了提高拓扑细化的效率,很多人提出了有大量的优化算法。Ma等人提出并行细化算法,通过定义能够保留器官拓扑结构和连通性的删除模板,一次迭代能够同时检测并删除多个体素。Paik等人结合最短路径法,用并行细化方法提高计算效率。Manzanera等人提出方向细化算法,利用不同的方向和条件标定体素,每一步仅剥离特定方向上的体素。Palgyi等人根据Rosenfeld三维细化准则设计出一系列删除模板,在该算法中,凡满足任意模板的边界点均可在一次迭代过程中并行地被删除。但模板非常多,袁非牛采用一种位编码和坐标变换的方式简化了模板测试,提出一种基于查找表的快速导航路径提取算法。Robert等人建立了关于所有体素格局拓扑测试结果的查找表,根据此表用索引代替拓扑测试,来判断体素点是否可以删除,从而大大缩短拓扑细化的运行时间。但是本质上,拓扑细化算法对局部噪声和外表很敏感,并且他生成的路径有很多枝节,需要进一步处理才能使用。
  \item[距离变换] \hfill \\
首先将分割后的二值图像划分为内部体素、外部体素和边界体素三类。定义内部体素到边界体素的最小距离为关于该点的距离变换,然后将变换后的距离图看成是一个有向或无向的加权图,最后采用Dijkstra最短路径算法得到中心路径。相对于拓扑细化算法,该算法计算量较小,速度较快。但是由于没有保证拓扑结构,最后的路径可能存在打折,拐点,需要进一步处理才能使用。
  \item[势能场法] \hfill \\
将距离变换法中的欧拉距离图替换为对象域的连续势能场,场中的最大势能体素对应中心路径。
  \item[基于水平集法] \hfill \\
中心路径可以看做体数据中两点之间的能量最小代价路径。
  \item[基于切割的中心路径提取] \hfill \\
此类方法利用分割目标时的数据信息提取目标的中心路径。
\end{description}

现有方法存在的问题:
\begin{enumerate}
    \item 算法复杂,实现困难;
    \item 计算复杂度高;
  \item 对非专业的用户不友好,设置使用复杂;
\end{enumerate}

\section{课题研究内容}
\subsection{研究目标}
本项目研究用于人体大肠中心路径提取的高效易于实现的算法,为上层漫游系统提供漫游的路径。

\subsection{主要任务}

\begin{enumerate}
    \item 将原始的网格数据体素化为体数据,为中心路径提取提供基础;
    \item 设计并实现中心路进提取算法;
    \item 为中心路径提取部分实现简单的界面,使其可以独立出来成为一个小工具。
\end{enumerate}

\section{论文组织结构}
第一章“绪论”,对本论文所阐述的课题进行说明,主要包括课题的背景、国内外研究现状、课题研究内容等。

第二章“需求分析”,详细描述了系统的功能性需求和非功能性需求,同时对界面和运行环境的需求做了明确的解释。

第三章“相关原理与技术,介绍了论文中涉及的关键技术、关键算法等内容。包含对“快速行进法”、“增强快速行进法”等核心内容的介绍。

第四章“系统总体设计”,从宏观角度,全面地展现了整个系统的层次结构以及各个层次、各功能之间的联系,同时对系统进行了模块划分。

第五章“系统详细与实现”,以系统架构图为标准,对本人负责的中心路径提取功能模块进行了详细的阐述。

第六章“系统测试评估”,对本人负责的中心路径提取算法的测试过程结果进行了记录和分析。

\section{本章小结}
本章介绍了课题的背景与研究意义,同时对国内外的研究现状进行了充分的讨论。对于本课题的研究内容从“研究目标、主要内容、关键技术、预期成果”等方面进行了阐述。此外,对论文的组织结果进行了概要说明,以便对论文本身进行整体把握。
