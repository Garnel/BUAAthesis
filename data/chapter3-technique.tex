% !Mode:: "TeX:UTF-8"
\chapter{相关原理与技术}

\section{程函方程}
程函方程是当使用WKB理论来近似波动方程时在波动传播问题中碰到的非线性偏微分方程。它从电磁学的麦克斯韦尔方程组导出,并在物理 (波动)光学和几何 (射线)光学之间起连接作用。

它的一般形式是:
\begin{equation}
    \label{eikonal_equation_1}
    \left| \nabla u(x) \right| = a(x), x \in \Omega
\end{equation}

约束条件是边界$u(x)$梯度为0。$a(x)$值为正。$\nabla$是梯度,$\left| \right|$是欧几里得范数。这里,右边的$a(x)$是已知的,物理上来说,$u(x)$的求解结果是边界运动到里面某个点$x$所需要的最短时间,此时$a(x)$表示在$x$上花费的时间。

一种快速计算程函方程近似解的方法就是快速行进法。当$a = 1$的特殊情况下,方程的解就是内部点到边界的有符号距离。

在2D情况下,我们求解程函方程,即下面的问题:
\begin{equation*}
    \label{eikonal_equation_2}
    \left\{
    \begin{aligned}
    \left| \nabla u \right| = a & \mbox{on} & \Omega \\
    u = 0 & \mbox{on} &  \partial\Omega
    \end{aligned}
    \right.
\end{equation*}

$\Omega$是平面里面的一个封闭物体。我们离散化变量$x$:$x_{I} = I\Delta x\mbox{其中}I = (I_{1}, I_{2}) \in Z_{2}, \Delta x > 0$。然后将
\begin{equation*}
    \label{eikonal_equation_3}
    \left| \nabla u(x) \right| - a(x) = 0
\end{equation*}

离散化为以下格式:
\begin{equation*}
    \label{scheme}
    S_{I}(\{u_{J}\}_{J \in V(I)}) = 0
\end{equation*}

其中,$V(I) = \{J \in Z^{2}, \left| J - I \right| \leq 1\}$,也就是说$V(I)$是点$I$在平面网格上的相邻5点(看)。于是我们有了
\begin{equation*}
    \label{eikonal_equation_3}
    V(I) = \{I, I^{1, +}, I^{1, -}, I^{2, +}, I^{2, -}\} \mbox{ 其中 } I^{\alpha, \pm}  = I \pm e_{\alpha}, \alpha = 1, 2
\end{equation*}

我们推荐Rouy-Tourin格式
\begin{equation}
    \label{eikonal_equation_3}
    \begin{aligned}
    & -a(x_I) + \sqrt{(\max(0, \frac{u_I - u_{I^{1, -}}}{\Delta x}, \frac{u_I - u_{I^{1, +}}}{\Delta x}))^2 + (\max(0, \frac{u_I - u_{I^{1, -}}}{\Delta x}, \frac{u_I - u_{I^{1, +}}}{\Delta x}))^2} \\
    & := S_I(u_I, u_{I^{1, -}}, u_{I^{1, +}}, u_{I^{2, -}}, u_{I^{2, +}}) := S_I[u]
    \end{aligned}
\end{equation}

\section{快速行进法}
\subsection{数学证明}
快速行进法(Fast Marching Method, FMM)是由Sethian在1996年提出的,用来求解从已知边界向内部演化的过程。它以某种方式记录了物体边界向前行进的信息。
\begin{figure}[h!]
    \centering
    \includegraphics[width=300bp]{figure/sets_fmm.png}
    \caption{FMM的各个集合:已知点集$A^n$,前锋点集$F^n$和未知区域}
    \label{fig-sets_fmm}
\end{figure}

首先我们定义一个递增时间序列$(t_{n})_{n}$,其中$t_{0} = 0$,一个递增集合序列$(A^{n})_{n}$和一个非递增函数序列$(u^{n})_{n}$,其中,每一个$Z^{2}$网格上的点$I$都对应定义$u^{n} = (u^{n}_{I})_{I}$。$u^n$在$A^n$上面计算出来的值就是程函方程里$u$的解。这也是为什么$A^n$被称为已知点集,因为$u^n$与$A^n$外面的点没有关系,我们设置
\begin{equation*}
    \label{far_region}
    u^n_I = +\infty \mbox{当} I \in Z^2 \setminus A^n
\end{equation*}

另一方面,$t_n$的值是
\begin{equation*}
    \label{tn_in_fn}
    u^n_I = t_n \mbox{当} I \in A^2 \setminus A^{n-1}
\end{equation*}

即$t_n$是已知点集$A^n$中相对于$A^n$新增的点的一般值。

看图很容易就能想到,前锋点集$F^n$就是已知点集$A^n$的离散边界,就是说
\begin{equation*}
    \label{fn_1}
    F^n = \{ I \in Z^2 \setminus A^n, \mbox{所以} V(I)\cap A^n \neq \textrm \O \}
\end{equation*}

为了更方便的应用到FMM算法里面,我们改写上式为
\begin{equation*}
    \label{fn_2}
    F^n = ( \underset{I \in A^n}{\cup}V(I)) \setminus A^n
\end{equation*}

前锋点集$F^n$是在$A^n$外面,离$A^n$很近的一组网格点。所以我们又称前锋点集为“窄带”。我们将要在这个点集里面为$A^{n+1} \setminus A^n$寻找新的点,即
\begin{equation*}
    \label{fn_3}
    A^{n+1} \setminus A^n \subset F^n
\end{equation*}
$A^n \cup F^n$的补集就被成为未知区域,因为这些点在第$n$步到第$n+1$步都不会被用到。

重要的是确定$F^n$里面哪些点可以被包含到$A^{n+1} \setminus A^n$里面。为了达到这个目的,我们先对每个前锋点集$F^n$里面的点$I$计算一个值$\widetilde{u}^n_I$。这个值是$u_I$的一个候选值(可能比$u_I$更大)。在这些值里面,我们只取最小的那个值
\begin{equation*}
    \label{tn+1}
    t_{n+1} = \underset{I \in F^n}{\min}\widetilde{u}^n_I
\end{equation*}
这些点就是新的已知点。也就是说
\begin{equation*}
    \label{tn+1}
    A^{n+1} \setminus A^n = \{ I \in F^n, \widetilde{u}^n_I = t_{n+1}\}
\end{equation*}
这恰好就解释了函数$u$的值就是从$\Omega$边界到达点$x_I$所花费的最短时间。我们定义新的$u^{n+1}$
\begin{equation*}
    \label{un+1}
    u^{n+1}_I = \left\{
    \begin{aligned}
    & t_{n+1} & I \in A^{n+1} \setminus A^n \\
    & u^n_I & \mbox{其他}
    \end{aligned}
    \right.
\end{equation*}

那么怎么计算这个$\widetilde{u}^n_I$的值呢?
对于给定的函数$u^n$,我们只需要找到下面方程的解$\widetilde{u}^n_I$。
\begin{equation}
    \label{solve_un}
    S_I(\widetilde{u}^n_I, \{u^n_J\}_{J \in V(I) \setminus \{I\}}) = 0
\end{equation}
要注意的是,如果$I$的相邻点$J$的值$u^n_J = +\infty$,这个点在计算的时候是不使用的,因为这种点在已知点集的补集里面,它们没有携带任何信息。我们只取在$A^n$里面的$J$点。例如图片\ref{closer_points},$A, B$在$A^n$里面,$C, D$不在$A^n$里面($C$在前锋上,$D$在未知区域)。所以$C, D$在计算$\widetilde{u}^n_I$是不会用到的。只有$J = A, B$对应的$u^n_J = u_J$会被用来计算$\widetilde{u}^n_I$。
\begin{figure}[h!]
    \centering
    \includegraphics[width=150bp]{figure/closer_points.png}
    \caption{$I$点相邻点}
    \label{closer_points}
\end{figure}

接下来我们看一下快速行进法的算法描述:

初始化
\begin{equation*}
    \label{fmm_init}
    \left\{
    \begin{aligned}
    & t_0 & = & 0 \\
    & A^0 & = & \{I \in Z^2, x_I \notin \Omega\} \\
    & u^0_I & = & \left\{
        \begin{aligned}
        & 0 & I \in A^0 \\
        & +\infty & I \notin A^0
        \end{aligned}
        \right.
    \end{aligned}
    \right.
\end{equation*}

从第$n$步到第$n+1$步,我们假设$t_n, A^n, u^n$都已知。

首先,我们计算时间的候选值$\widetilde{u}^n_I$,对于每一个$I \in F^n$,找到$\widetilde{u}^n_I$的唯一解:
\begin{equation*}
    \label{uI_unique}
    0 = S_I(\widetilde{u}^n_I, \{u^n_J\}_{J \in V^*(I)}), \mbox{其中} V^*(I) = V(I) \setminus \{I\}
\end{equation*}

然后,取最小的候选值。
\begin{equation*}
    \label{min_tn+1}
    t_{n+1} = \underset{I \in F^n}{\inf} \widetilde{u}^n_I
\end{equation*}

最后,更新点集和各个函数值。新的已知点集为
\begin{equation*}
    \label{new_an+1}
    NA^{n+1} = \{I \in F^n, \widetilde{u}^n_I = t_n\}
\end{equation*}
可得
\begin{equation*}
    \label{new_values}
    \left\{
    \begin{aligned}
    & t_{n+1},\\
    & A^{n+1} & = & A^n \cup NA^{n+1}, \\
    & u^{n+1}_I & = & \left\{
        \begin{aligned}
        & t_{n+1} & I \in NA^n \\
        & u^n_I & \mbox{其他}
        \end{aligned}
        \right.
    \end{aligned}
    \right.
\end{equation*}

算法结束。

\subsection{伪代码}
这里我们考虑方程\ref{eikonal_equation_1}的特殊情况:$a = 1$,即
\begin{equation}
    \label{special_eikonal_equation}
    \left| \nabla u(x) \right| = 1
\end{equation}
其中$u = 0$在物体的边界上。这个时候的$u$是物体边缘到内部点的距离的很好的近似。
快速行进法在算法上很容易实现,对于每一个2D网格点或者坐标为$(i, j)$的像素点,我们存储它的距离变换的浮点数值$T_{ij}$和一个标志$f_{ij}$。这个标志可能有三个值:
\begin{itemize}
\item BAND: 表示点属于当前的前锋点集或者说窄带($F^n$),他的$T$值依然在变化中。
\item INSIDE: 表示点在移动前锋内部的未知点集,它的$T$值目前未知。
\item KNOWN: 表示点在移动前锋后面(已知点集$A^n$),它的$T$值已知。
\end{itemize}

初始化如代码\ref{fmm_init_code}所示,MAX\_VALUE表示比任何可能的$T$都要大的值,比如说$10^6$。
\begin{lstlisting}[
    language={C},
    caption={快速行进法初始化代码},
    label={fmm_init_code},
]
for all points (i,j)
{
    if ((i,j) on initial boundary)
    {
        f(i,j)=BAND; T(i,j)=0;
        add (i,j) to NarrowBand;
    }
    else if ((i,j) inside boundary)
    {
        f(i,j)=INSIDE; T(i,j)=MAX_VALUE;
    }
    else /* (i,j) outside boundary */
    {
        f(i,j)=KNOWN; T(i,j)=0;
  }
}
\end{lstlisting}

演化时,我们将方程\ref{special_eikonal_equation}离散近似为:
\begin{equation}
    \label{discretized_eikonal_equation}
    \max(D^{-x}T, -D^{+x}T, 0)^2 + \max(D^{-y}T, -D^{+y}T, 0)^2 = 1
\end{equation}
其中$D^{-x}T(i, j) = T(i, j) - T(i - 1, j)$,并且$D^{+x}T(i, j) = T(i + 1, j) - T(i, j)$,$y$轴方向类似。方程\ref{special_eikonal_equation}可以由循环在每个网格点求解方程\ref{discretized_eikonal_equation}得到结果,对于$N$个网格点时间复杂度是$O(N^2)$。如代码\ref{fmm_iteration_code}所示。
\begin{lstlisting}[
    language={C},
    caption={快速行进法迭代代码},
    label={fmm_iteration_code},
]
while (NarrowBand not empty)
{
    P(i,j)=head(NarrowBand); 						/* STEP 1 */
    remove P from NarrowBand;
    f(i,j)=KNOWN;
    for point (k,l) in {(i−1,j),(i,j−1),(i+1,j),(i,j+1)}
        if (f(k,l)!=KNOWN)
        {
            if (f(k,l)==INSIDE) f(k,l)=BAND; 	    /* STEP 2 */
            sol=MAX_VALUE; 							/* STEP 3 */
            solve(k−1,l,k,l−1,sol);
            solve(k+1,l,k,l−1,sol);
            solve(k−1,l,k,l+1,sol);
            solve(k+1,l,k,l+1,sol);
            T(k,l)=sol;
            insert (k,l) in NarrowBand; 			/* STEP 4 */
        }
}

solve(int i1,int j1,int i2,int j2,float& sol)
{
    float r,s;
    if (f(i1,j1)==KNOWN)
        if (f(i2,j2)==KNOWN)
        {
            r = sqrt((2−(T(i1,j1)−T(i2,j2))*(T(i1,j1)−T(i2,j2)));
            s = (T(i1,j1)+T(i2,j2)−r)/2;
            if (s>=T(i1,j1) && s>=T(i2,j2)) sol=min(sol,s);
            else
            {
                s += r;
                if (s>=T(i1,j1) && s>=T(i2,j2)) sol=min(sol,s);
            }
        }
        else sol=min(sol,1+T(i1,j1));
    else if (f(i2,j2)==KNOWN) sol=min(sol,1+T(i1,j2));
}
\end{lstlisting}

\section{增强快速行进法}
快速行进法计算的是内部点到边界的距离,这里介绍一下增强的快速行进法用来产生骨架。图表\ref{skel_overview}概述了算法应用到一个锯齿状矩形的整个过程。很容易想到骨架点总是由紧凑的边界段在前锋行进时重合产生的。这个方法的重点就在于确定每一个行进前锋上的点是由边界上哪个点得来,为了实现这个目的,我们为每个网格点新增了一个实数属性$U$。初始化时,我们选取任意一个边界点,设置$U=0$,从这个点开始,我们为每个边界点设置一个单调递增的$U$值,这个$U$值等于这个点沿边界到初始点的长度。所以U是边界上两点沿边界的距离参数化的结果。有个例外就是$U$在各自的连通边界上开始演化,这个我们后面再讲。
\begin{figure}[h!]
    \centering
    \includegraphics[height=200bp]{figure/skel_overview.png}
    \caption{骨架提取算法预览}
    \label{skel_overview}
\end{figure}

