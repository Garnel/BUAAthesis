% !Mode:: "TeX:UTF-8"
\chapter*{总结与展望}
\addcontentsline{toc}{chapter}{总结与展望}
\section*{总结}
本文侧重研究3D物体中心路径提取算法的研究和实现。扩展了传统快速行进法的使用范围,用传统的快速行进法实现了3D物体中心路径的提取。本文算法原理简单,易于实现,用户学习成本低,体验良好。

具体工作:
\begin{enumerate}
    \item 阅读了大量相关文献,对于虚拟漫游概念,中心路径提取技术有了一定了解;
    \item 设计实现了一种很简单的网格数据体素化的方法;
    \item 深入理解了快速行进法的原理,并运用实现了3D物体中心路径的提取。
\end{enumerate}

\section*{展望}
由于研究时间和条件所限,本文介绍的算法还需要进一步完善。根据实验结果,本文算法可以从以下几个方面进行改进,以便能够达到使用目的:
\begin{enumerate}
    \item 对于分辨率较低的数据,体素化时可以通过插值等方法将不连通的边界连起来;
    \item 对于分辨率低的体数据生成的中心路径不连通的情况,可以通过放大体数据等方法解决,或者在求交集过程中改进算法;
    \item 提高系统运行效率,体素化和计算路径都是计算量大的任务,后续需要改进算法,提高运行速度,缩短运行时间。
\end{enumerate}

\cleardoublepage
