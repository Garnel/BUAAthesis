% !Mode:: "TeX:UTF-8"
% 任务书中的信息
%% 原始资料及设计要求
\assignReq
{人体大肠网格数据}
{快速行进法增强快速行进法}
{}
{}
{}
%% 工作内容
\assignWork
{将原始的网格数据体素化为体数据,为中心路径提取提供基础;}
{设计并实现中心路进提取算法;}
{为中心路径提取部分实现简单的界面,使其可以独立出来成为一个小工具。}
{}
{}
{}
%% 参考文献
\assignRef
{高向军, 杨克领. 虚拟内窥镜的路径规划算法研究 [J]. 计算机应用研究, 2011, 28(3).}
{Telea A., Van Wijk J. J. An augmented fast marching method for computing }
{skeletons and centerlines[A]. Proceedings of the symposium on Data }
{Visualisation 2002[C]. .[S.l.]:[s.n.] , 2002:251–ff.}
{Monneau R. Introduction to the Fast Marching Method[J]. 2010.}
{朱维松. 基于距离变换的纤维骨架提取算法研究 [D].[S.l.]: 东华大学, 2008.}
{刘俊涛, 刘文予, 吴彩华, et al. 一种提取物体线形骨架的新方法 [J].}
{自动化学报,2008, 34(6):617–622.}
